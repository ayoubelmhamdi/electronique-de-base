\documentclass[twocolumn,french]{article}
\usepackage[top=2.0cm, bottom=2.5cm, left=1.0cm, right=1cm]{geometry}
\usepackage[T1]{fontenc}
\usepackage[utf8]{inputenc}
\usepackage{lmodern}
\usepackage{ntheorem}
\usepackage{babel}
\usepackage{multicol}
\usepackage{multirow}
\usepackage{fancyhdr}
\usepackage{amsfonts}
\usepackage{amsmath}
\usepackage{amssymb}
\usepackage{latexsym}
\usepackage{array}
\usepackage{graphicx}
%%%%%%%%%%%%%%-----NEW COMMAND-------%%%%%%%%%%%%%%%%%%%%%%%%%%%%%%%%%%
\newcommand{\dis}{\displaystyle}
\newcommand{\ve}{ \overrightarrow}
% \mathchardef\times="2202
\newcommand{\C}{\mathbb{C}}
\newcommand{\R}{\mathbb{R}}
\newcommand{\Q}{\mathbb{Q}}
\newcommand{\Z}{\mathbb{Z}}
\newcommand{\N}{\mathbb{N}}

%%%%%%%%%%%%%%----Ayoub----%%%%%%%%%%
% \DeclareUnicodeCharacter{2212}{-}
\DeclareMathOperator{\e}{e}
\usepackage{siunitx}
\usepackage{xcolor}


%%%%%%%%%%%%%%%%%%%%%%%%%%%%%%%%%%%%%%%%%%%%%%%%%%%%%%%%%%%%%%%%%%%
%-----------------Header and footer-------------------------
\pagestyle{fancy}
\renewcommand{\headrulewidth}{1,5pt}
\renewcommand{\footrulewidth}{1 pt}
\fancyhead[CO,LE]{\textbf{ELECTONIQUE TRPM\\}}
\fancyfoot{} % clear all footer fields
\fancyfoot[LE,RO]{\textbf{E.MHAMDI}}
\fancyfoot[LO,CE]{\textbf{TRPM MEKNES}}
\fancyfoot[CO,RE]{\thepage}
\renewcommand{\headrulewidth}{0.4pt}
\renewcommand{\footrulewidth}{0.4pt}
\setlength{\columnseprule}{0.4pt}% the line between colomn paper

%%%%%%%%%%%%%%%%%%%%%%%%%%%%%%%%%%%%%%%%%%%%%%%%
%%%%%%%%%%%%%%---exercices style------%%%%%%%%%%%%%%%%%%%%%%
\theoremstyle{plain}
\theorembodyfont{\normalfont}
\theoremseparator{~--}
\newtheorem{exo}{Exercice}%[section]


%%%%%%%%%%%%%%%%%%% ----starting the body ------%%%%%%%%%%%%
\begin{document}

\color{black}
%%%%%%%%%%-----------exo 1%%%%%%%%%%%%%%%%%%%%%%%%%%%%%%%%%%%%%%%%%%%%%
\begin{exo}
\textit{\textbf{ LE GAP:}} \\
On considere l'arsenuire de gallium $G_aA_s$ pour lequel a la tempertature
ambiante $T_0=300K$.\\
On done
\begin{itemize}
	\item[$\blacksquare$] $h = 6.62\ 10^{-34}\ j.s$
	\item[$\blacksquare$] $c = 3.00\ 10^8\ m.s^{-1}$
	\item[$\blacksquare$] $m_0 = 0.91\,10^{-30}\ kg$
  \item[$\blacksquare$] $N_0 = 2.5\ 10^{25}\ m^{-3}$
  \item[$\blacksquare$] $\mu_{n^{(T_0)}} = 0.80\ m^2.V^{-1}.s^{-1}$
  \item[$\blacksquare$] $\mu_{p^{(T_0)}} = 0.04\ m^2.V^{-1}.s^{-1}$
  \item[$\blacksquare$] $N_{c^{(T_0)}} = 4.5\ 10^{23}\ m^{-3}$
  \item[$\blacksquare$] $N_{v^{(T_0)}} = 7.0\ 10^{24}\ m^{-3}$
  \item[$\blacksquare$] $\xi = 0.964\ 10^{-10}\ F.m^{-1}$
\end{itemize}
On suppose que les mobilites sont independantes des densites 
impuretes presentes dans le semiconducteur et qu'elles evoluent
proportinnet a $T^{\frac{3}{2}}$

\begin{enumerate}
  \item[] pour un semi intrinsique $G_aA_s$ on a les 4 
   propiete suivant:
  \begin{align*}
    &n_i^2=p\cdot n                       &  &n_i^{int}=n=p            \\
    &n=N_c\e^{\frac{\xi_f-\xi_c}{2KT}}    &  &p=N_v\e^{\frac{\xi_v-\xi_f}{2KT}} 
  \end{align*}
\item Monter que le gap en fontion de tempertature $E_g=f(T)$ est :
    \begin{equation}
      E_g=KT\ln \frac{N_cN_v}{2}
      \label{eq:1}
    \end{equation}
    
\end{enumerate}


\end{exo}
\hrule


%%%%%%%%%%--------exo--------%%%%%%%%%%%%%%%%%%%%%%%%%%%%%%%%%%%%%%%%



\end{document}
